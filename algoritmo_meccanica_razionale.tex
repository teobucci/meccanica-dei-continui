\documentclass{article}
\usepackage{enumitem}
\usepackage[italian]{babel}
\usepackage[utf8]{inputenc}
\usepackage[margin=1.5in]{geometry}

\title{Algoritmo per risolvere esercizi di Meccanica Razionale}
\date{}
\author{di Filippo Cipriani, trascritto da Teo Bucci}

\begin{document}

\maketitle

\paragraph{Statica}

\begin{enumerate}
    \item PSP, scomodo, con le coppie non funziona.
    \item PLV, migliore, specialmente se hai sia dinamica che statica, quindi devi fare comunque la cinematica.
\end{enumerate}

\paragraph{Dinamica}

In tutti gli esercizi di dinamica serve l'energia cinetica (quindi la prima parte obbligatoria è di \textit{studio cinematico} per trovare le velocità), poi, trovata l'energia cinetica dipende dal numero di Coordinate Libere (CL) e dalla richiesta.

\begin{enumerate}
    \item Ti chiede il moto:
        \begin{enumerate}
            \item $1$ CL: teorema dell'energia cinetica\footnote{È solo un prodotto scalare tra forza e velocità del punto d’applicazione, non tiene conto di rotazioni e le uniche forze sono forza elastica e forza peso. Le tensioni, come per il PLV, non vengono considerate.} $\dot{T} = \Pi$.
            \item $2$ CL: non scappi dalla lagrangiana.\footnote{Se il potenzial dipende da $1$ CL, usi il momento cinetico, se dipende da $2$ CL, fai due lagrangiane (caso molto raro).}
        \end{enumerate}
        Se hai molle avrai un moto armonico e quindi sai già che verrà l'equazione brutta, sempre della forma $\ddot{q} + \omega ^{2} q = k$. \label{item-one}
    \item Ti chiede $\Delta $asta/qualcosa quando un'altra cosa raggiunge un certo valore: al 90\% qui devi usare il momento cinetico, calcoli $T$ e $U$, $U$ dipende da una sola coordinata (quando c'era quella richiesta non è mai capitato che $U$ dipendesse da $2$ CL), trovi il momento cinetico e integri per passare ai delta.
    \item Ti chiede $v_{f}$ o $\dot{\vartheta}_{f}$: conservazione dell'energia meccanica + lagrangiana.
    \item Ti chiede la tensione durante il moto: la trovi con la risultante sulla massa attaccata al filo, questa dipenderà sicuramente dalla derivata seconda di una coordinata libera, che dovrai trovare con il punto \ref{item-one}, poi equazioni cardinali.
    \item Ti chiede reazioni vincolari in un certo istante o il minimo coefficiente d'attrito per il disco: usi le equazioni cardinali, ma prima dovrai trovare il moto come nel punto \ref{item-one} perché dipendono quasi sempre da quello.\footnote{C'è un caso che si ripete spesso con punto con massa $m$ che scorre su lamina senza massa attaccata a centro di un disco (e chiede una reazione vincolare agente sulla lamina): lì è sempre meglio staccare il punto, calcolare la reazione di contatto tra punto e lamina e fare il momento della lamina rispetto al punto del centro del disco.}
\end{enumerate}

\end{document}